\documentclass{article}
%
\usepackage{multicol}
\usepackage{polski}
\usepackage[utf8]{inputenc}
\usepackage{bridge_macros}
\setlength{\textwidth}{7in}
\setlength{\textheight}{9in}
\setlength{\oddsidemargin}{-.4in}
\setlength{\topmargin}{-.5in}
%
\begin{document}
\section{Pierwsza iteracja}
\subsection{Ekrany}
\begin{itemize}
	\item Ekran ustawień, w którym osoba może wybrać np. nickname
	\item Możliwość wyboru gry solo lub przez bluetooth
	\item Wybór decku i możliwość dowolnego customizowania go (np. gra 2 taliami i dodatkowo 10 asów pik)
	\item Przy grze bluetooth, rozróżnienie na server i client
		\begin{itemize}
			\item Server - wchodzi na stół i czeka
			\item Client - dostaje listę stołów serwerów, które widzi. Po połączeniu dołącza do stołu
		\end{itemize}
	\item 
		
\end{itemize}

\subsection{Stół}
\begin{itemize}
	\item Możliwość dociągania kart z decku, przeciągania karty na stół i z powrotem do ręki,
		odwracania ich
	\item Serwer ma możliwość ponownego rozdania
	\item Wyświetlanie, kto jest obecny przy stoliku
\end{itemize}


\section{Ostateczny rezultat}
\subsection{Ekrany}
\begin{itemize}
	\item Dodatkowo możliwość gry online (załóż stół, dołącz do stołu)
	\item Możliwość wyboru gry
	\item Możliwość wyboru nie tylko kart, ale i rodzaju stołu (np. stół, gdzie można kłaść karty tylko na środek)
	\item Instrukcje do każdej gier
	\item Gra z komputerem
	\item Zakupy (skórki, nowe gry)
	\item Historia gier, rankingi
	\item ...
		
\end{itemize}

\subsection{Stół}
\begin{itemize}
	\item Możliwość dokładania kart, wyrzucania na discard
	\item Czat
	\item Historia zagrań
	\item ...
\end{itemize}

\end{document}
