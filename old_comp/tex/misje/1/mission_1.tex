\documentclass[10pt,a4paper]{protocol}

% Change the page layout if you need to
\geometry{left=1cm,right=9cm,marginparwidth=6.8cm,marginparsep=1.2cm,top=1cm,bottom=1cm}

% Change the font if you want to.

% If using pdflatex:
\usepackage[utf8]{inputenc}
\usepackage[T1]{fontenc}
\usepackage[default]{lato}

% If using xelatex or lualatex:
% \setmainfont{Lato}

% Change the colours if you want to
\definecolor{Bright}{HTML}{58318F}
\definecolor{Green}{HTML}{319866}
\definecolor{Black}{HTML}{111111}
\definecolor{LightGrey}{HTML}{515c50}
\colorlet{heading}{Green}
\colorlet{accent}{Bright}
\colorlet{emphasis}{Black}
\colorlet{body}{LightGrey}

% Change the bullets for itemize and rating marker
% for \risk if you want to
\renewcommand{\itemmarker}{{\small\textbullet}}
\renewcommand{\ratingmarker}{\faSpinner}

%% sample.bib contains your publications
\addbibresource{sample.bib}

\begin{document}
\name{Misja 1}
\tagline{Kryptonim 'Impreza u Ryśka'}
\logo{5.5cm}{"abc2"}



%% Make the header extend all the way to the right, if you want.
\docinfo{%
  Wbić na domówkę nieznajomych
  % Deadline: Niedziela 18.11.2018 godzina 23:59
}

\purpose{
	\begin{itemize}
		\item Drewno
		\item Pasztecik
	\end{itemize}
} % add a short discription of the purpose for this protocol
%% Provide the file name containing the sidebar contents as an optional parameter to \need.
%% You can always just use \marginpar{...} if you do
%% not need to align the top of the contents to any
%% \need title in the "main" bar.
\begin{fullwidth}
\makeheader
\end{fullwidth}

\need[materials]{Plan}

\step{1}{Znaleźć grupkę wyglądającą na taką, co doprowadzi nas do celu}{50}
\begin{itemize}
	\item Szukamy na mieście
	\item Szukamy w parku
	\item Szukamy w Macu
	\item Szukamy na sympatia.pl
\end{itemize}
\divider

\step{2}{Niepostrzeżenie poznać imię organizatora}{10}
\begin{itemize}
	\item Hipnoza
	\item Szantaż
	\item Czytanie w myślach
	\item Zapytać
\end{itemize}
\divider

\step{3}{Dotrzeć pod docelowy adres}{5}
\begin{itemize}
	\item Iść z grupą
	\item Śledzić
	\item Wyciągnąć w punkcie 2 i pojechać rowerem
	\item Zadzwonić do organizatora z prośbą, by po nas wyszedł
\end{itemize}
\divider

\step{4}{Wbić na imprezę}{20}
\begin{itemize}
	\item Skorzystać z wcześniej zakupionych ciastek
	\item Użyć zdobytych informacji, aby podszyć się pod znajomego z przedszkola
	\item Powiedzieć, że jest za głośno i zaraz wezwiemy policję, jeśli nas nie wpuszczą
	\item Wejść niepostrzeżenie razem z grupą
\end{itemize}
\divider

\step{5}{Zrobić furorę}{0}
\begin{itemize}
	\item Śmieszne żarty
	\item Sztuczki karciane
	\item Oświadczyny
	\item Opowieści o naszej głupocie
\end{itemize}
\divider
\begin{tikzpicture}[remember picture,overlay]
	\node[xshift=10mm,yshift=-60mm,anchor=north west] at (current page.north west){%
	\includegraphics[angle=30,width=200mm]{../secret}};
\end{tikzpicture}

\clearpage


\end{document}
