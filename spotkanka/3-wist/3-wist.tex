\documentclass{article}
%
\usepackage{bridge_macros}
\usepackage{tikz}

\newcommand\encircle[1]{%
  \tikz[baseline=(X.base)] 
    \node (X) [draw, shape=circle, inner sep=0] {\strut #1};}

%
\begin{document}
\section*{Ogólne zasady obrony i wistowania}
\begin{itemize}
	\item Dopóki brak dużo lepszych alternatyw - wistuj w kolor partnera.
	\item Z sekwensu (co najmniej 3 karty po kolei z honorem).
	\item Najlepiej w nielicytowane przez przeciwników.
	\item 'Trzecia ręka bije i płacze' - na trzeciej ręce prawie zawsze opłaca się położyć najwyższą kartę.
	\item Z kolei na drugiej kładziemy najniższą - partner zazwyczaj ma jakieś wsparcie.
	\item 'Figur na figur' - jeśli rozgrywający zagrywa figurę, a my mamy wyższą, opłaca się ją zagrać.
\end{itemize}
\subsection*{Do gry w bez atu}
\begin{itemize}
	\item Jak można, to w kolor partnera.
	\item Wist w najdłuższy kolor - chcemy sobie wyrobić jak najwięcej lew.
	\item Przy równej długości sugerujemy się tym, gdzie potrzebujemy mniej od partnera.
	\item Z sekwensów dwukartowych lepiej blotką, niż honorem.
\end{itemize}
\subsection*{Do gry w kolor}
\begin{itemize}
	\item Jak mamy w jakimś kolorze asa i króla wistujemy w ten kolor bez wahania.
	\item Jak można, to w kolor partnera.
	\item Zakazane są wisty spod waleta i spod asa.
	\item Nie wistujemy w atuty.
	\item Z sekwensów dwukartowych lepiej honorem, niż blotką.
	\item Wskazane wisty z \textbf{singla} (jedna karta w kolorze), może partner ma asa, odegra nam i weźmiemy atutem.
	\item Czasem też z dubla.
\end{itemize}
\newpage
\section*{Szczegółowe zasady wistu odmiennego}
Wist, którym będziemy się posługiwać nosi nazwę \textbf{odmiennego} i jest to najpopularniejszy system wistowy
w Polsce. Składa się z kilku prostych zasad:
\begin{itemize}
	\item Z dwóch blotek - niższą.
	\item Z honoru i blotki - wyższą.
	\item Z trzech kart - środkową.
	\item Z czterech i więcej kart z honorem - czwartą od góry.
	\item Z czterech i więcej kart bez honoru - drugą od góry.
\end{itemize}
Przykłady (H - honor, tj. (A)s, (K)ról, (D)ama lub (W)alet, x - blotka tj. karta niższa od 10):
\begin{multicols}{4}
\setlength{\parindent}{0pt}
x\encircle{x}\\
\encircle{H}x\\
\vfill\null
\columnbreak
x\encircle{x}x\\
H\encircle{x}x\\
\encircle{K}Dx\\
A\encircle{W}x\\
\vfill\null
\columnbreak
x\encircle{x}xx\\
Hxx\encircle{x}\\
\encircle{K}DWx\\
\encircle{K}D10x\\
\encircle{K}Dx\encircle{x}\\
\encircle{D}W9x\\
D\encircle{10}9x\\
KWx\encircle{x}\\
K\encircle{W}10x\\
\encircle{W}109x\\
\vfill\null
\columnbreak
Hxx\encircle{x}x\\
x\encircle{x}xxx\\


\end{multicols}
\end{document}
